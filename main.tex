\documentclass{article}
\usepackage{graphicx} % Required for inserting images
\usepackage{fancyhdr}
\usepackage{amssymb}
\usepackage{amsmath}
\usepackage{mathtools}
\usepackage{amsthm}
\usepackage{hyperref}
\usepackage{blindtext}
\usepackage[margin=1in,footskip=0.25in]{geometry}
\usepackage[most]{tcolorbox}


% if you ever want to ente% Required for inserting images

\title{Analysis Notes 131AH}
\author{OMKAR TASGAONKAR}
\date{Winter Quarter - January-March 2026}
\newtheorem{thm}{Theorem}[section]
\newtheorem{lem}{Lemma}[section]
\newtheorem{defn}{Definition}[section]
\newtheorem{cor}{Corollary}[section]
\newtheorem{axm}{Axiom}[section]

\tcolorboxenvironment{thm}{
  colback=lime,
  colframe=black,
  breakable
}

\tcolorboxenvironment{lem}{
  colback=yellow,
  colframe=black,
  breakable
}

\tcolorboxenvironment{defn}{
  colback=pink,
  colframe=black,
  breakable
}

\tcolorboxenvironment{cor}{
  colback=pink,
  colframe=black,
  breakable
}

\tcolorboxenvironment{axm}{
  colback=white,
  colframe=black,
  breakable
}


\begin{document}

\maketitle
\section*{If you are reading this...}
Hello! These are my notes for analysis, which I have typed up to save to github so I can reference them in the future. Things will be written in my own words and may not be fully correct, but this is just my attempt to fully internalize everything and write it down as precisely as possible.
\tableofcontents
\pagebreak

\section{Propositional and First-Order Logic}
\begin{defn}
    A proposition is a statement that takes a TRUE or FALSE value
\end{defn}
For example, "birds are mammals" is a valid proposition, which takes on the value of FALSE. We also define a "primitive proposition", which is one with no connectives or quantifiers.\\

\begin{defn}
    A connective is a unary or binary operator allowing us to chain propositions to create new propositions
\end{defn}
The connectives we will work with are $\lnot$ (logical not), $\land$ (and), $\lor$ (or), $\implies$ (implies), and $\iff$ (bioconditional). The $\lnot$ operator is the only unary operator, which switches the value of the proposition. The others are binary, requiring two propositions and outputting one value.\\

We view an interesting truth table for $\implies$:
\begin{center}
\begin{tabular}{| c | c  | c |}
\hline
$P$ & $Q$ & $P \implies Q$ \\
\hline
T & T & T\\
T & F & F\\
F & T & T\\
F & F & T\\
\hline
\end{tabular}
\end{center}

\indent When $P$ is false but $Q$ is true, the entire statement is "vacuously" true as the expected proposition $Q$ is true no matter $P$. When both are false, the proposition is true as well. So in general when $P$ is false, $P \implies Q$ is true.\\

\begin{lem}
    $(P \implies Q) \iff \lnot(P \land \lnot Q) \iff \lnot P \lor Q$
\end{lem}
Here the biconditional means that the propositions are equivalent. We can use the biconditional to represent when two statements imply each other or are logically the same.\\

This lemma highlights proof by contradiction, where we assume $P$ and $\lnot Q$ and we show some contradiction (or show that the proposition is false). Then the logical not of the proposition is true. The second equivalence is an application of DeMorgan's Laws for logic.\\

\begin{lem}
    $(P \implies Q) \iff \lnot Q \implies \lnot P$
\end{lem}
This is the method of proof by contrapositive. This implication is called the \textbf{"only if"} direction, while the reverse is the \textbf{"if"} direction.\\

\begin{defn}
    We propose two quantifiers: $\forall$ (for all/for each) and $\exists$ (there exists)
\end{defn}
A first order statement assumes the form: (quantifier)(variable): (proposition). This is the basis of First Order Logic, an extension of propositional logic. For example:
\begin{equation*}
    \forall x : P(x)
\end{equation*}
translates to: for all x such that $P(x)$ is true. The terminology "some" is equivalent to $\exists$.\\

\begin{lem}
    $\lnot(\forall x : P(x)) \iff \exists x : \lnot P(x)$ and $\lnot(\exists x: P(x)) \iff \forall x : \lnot P(x)$
\end{lem}
Here we see how the logical not operator distributes over quantifier and proposition.\\

\begin{thm}
    $\forall x \forall y: P(x, y) \iff \forall y, \forall x: P(x, y)$ and $\exists x, \exists y: P(x, y) \iff \exists y, \exists x: P(x, y)$
\end{thm}
If the quantifiers for two variables are the same, we can switch the order of the quantifiers. This will appear in proofs involving sets and supremums/infimums. However, if they are not the same, then we cannot suggest an equivalence.\\

\begin{thm}
    $\exists x \forall y: P(x, y) \implies \forall y \exists x: P(x, y)$
\end{thm}
The reason why this is a one-way implication is because the left hand statement is more specific. We will encounter this when dealing with images and preimages. We take the following example:\\

\textbf{Mathematical Example}: If there exists one $x$ greater than all $y$, then for each $y$ there exists an $x > y$. In this case that $x$ is the same. For the converse, if each $y$ has an $x > y$, there may not be a singular $x$ greater than all $y$.\\

\textbf{Figurative Example}: If there is one building to which all apartments belong, then all the apartments belong to that building. However, if all the apartments belong to \text{a} building, not necessarily they belong to one building. We can have the apartments in multiple buildings.
\pagebreak

%------------------------------------------------------------%

\section{Naive Set Theory and Zermelo Fraenkel}
\subsection{Naive Set Theory}
Naive Set Theory was the initial proposed set theory. A set being a container for some sort of objects. To construct a set, we have the following:
\begin{defn}
    The Comprehension Principle: $A \coloneq \{x : P(x)\}$
\end{defn}
The symbol $\coloneq$ means "defined as". If we use "$=$" when referring to sets, that is not the same as using $\coloneq$.\\

The Comprehension Principle tells us that a set is some elements $x$ that satisfy a proposition $P(x)$. We won't dwell on this theory too much, but it has pitfalls such as Russell's paradox:
\begin{equation*}
    A \coloneq \{x : x \notin x\}
\end{equation*}

Here if $x \notin x$, then we could say $A \notin A$, but the proposition tells us that $A \in A$, so we come across a contradiction.\\

\subsection{Zermelo Fraenkel}
\begin{axm}
    \textbf{Axiom of Separation} Any set $A$ can be defined as $\{x \in U : P(x)\}$.
\end{axm}
We see that the elements $x \in A$ are being drawn from $U$ which in this case can be a universe (all the possible sets, but this is not itself a set), or a bigger set. By restricting elements, we can prevent self references such as those in Russell's paradox.\\

\begin{axm}
    \textbf{Axiom of Extensionality} $\forall A, B: A = B \iff \forall x: (x \in A \iff x \in B)$
\end{axm}
We mentioned previously that $\coloneq$ is not the same as $=$ for sets. Extensionality gives the definition of $=$ to be that the sets must have the same elements. The axiom itself is sometimes written as an implication rather than a biconditional.\\

\begin{axm}
    \textbf{Empty Set}: $\exists \varnothing : \forall x \in U: x \notin \varnothing$
\end{axm}
We postulate the existence of the empty set, which contains none of the possible elements in the universe. Now, how do we build the set of all subsets (aka the powerset)?\\

\begin{defn}
    $\subseteq$ (subset relation): $\forall A, B: A \subseteq B \implies \forall x \in A: x \in B$
\end{defn}
Let's try to use the Axiom of Separation and this definition to construct the powerset. We can say maybe for a set $A$ that $\mathcal{P}(A) \coloneq \{B \in C: B \subseteq A\}$. But this definition requires a set $C$ to exist in which $B$ belongs, so either this is the powerset or a set containing more elements than the powerset, and so we have a circular definition.\\

\begin{axm}
    \textbf{Powerset}: $\forall A \exists \mathcal{P}(A): (\forall B \subset A: B \in \mathcal(P)(A))$
\end{axm}
We have the same issue when we go to define a set such as $\{x, y\}$, which is called the pairset, as each element is also inherently a set.\\

\begin{axm}
    \textbf{Pairset}: $\forall x, y \exists A \forall Z \in A: x = z \lor y = z$
\end{axm}
The pairset allows us to show the existence of a singleton set $\{x\}$ if both the elements given are the same as $\{x, x\} = \{x\}$. This can be shown by extensionality.\\

\begin{axm}
    \textbf{Infinite Set}: $I \coloneq \{x: x \in I \implies \{x\} \in I\}$
\end{axm}
This is one possible construction of the infinite set we will use for the naturals; however, this axiom postulates the existence of infinite sets. Many of the constructions in analysis will hinge on this axiom; without it, we fail to construct the idea of "natural numbers" let alone "real numbers".\\
\pagebreak

%----------------------------------------------------------------------------%

\section{Ordered Pairs and Relations}




\end{document}
